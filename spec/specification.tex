\documentclass[11pt,a4paper]{article}
\usepackage[utf8]{inputenc}
\usepackage{amsmath}
\usepackage{amsfonts}
\usepackage{amssymb}
\usepackage{color}
\usepackage{listings}
\usepackage[linktocpage,colorlinks=true]{hyperref}
\usepackage{graphicx}
\usepackage{multicol}

\begin{document}

\title{PerfOCaml: a Benchmarking Environment for OCaml\\Specification 2014/05/27}

\maketitle

\section{Use Cases}

Some examples of what users should be able to do with PerfOCaml.

\subsection{Compiler Patch Writer}

\subsubsection{Local Testing}

After building a modified compiler (with the right {\tt --prefix}
option), install it, build and run different benchmarks, compare it
with previous runs and previously generated reference results.

\subsubsection{Publishing}

Publish a branch on {\tt github}, register it on the site (maybe
through a pull request). A set of benchmarks are run and comparison
results are displayed on the public site.

\subsection{Compiler Core Team}

Regular snapshots of OCaml trunk and branches run against a series of
benchmarks.

\subsection{Library Developer}

Provides benchmarks decriptions for its own library to be used when
testing the compiler and for comparing library versions.

\subsection{External Analysis}

\begin{itemize}
\item
  Publicly available simple informations about compiler overall
  performances evolution.

\item
  Go to bench web site, download some performance analysis,
  comment/summit bug report according to the data to OCaml or to a
  library.

\end{itemize}


\subsection{User in a difficult to reproduce scenario}

If a user have access to a computer that have difficult to reproduce
specificites (exotic architecture, system, ...), he sould be able to
submit itself some benchmark results associated with system
informations.

\section{Proposition}

Basic aim: minimize\footnote{Fabrice: we should not focus on
  minimizing changes to OPAM, but on making the project as ambitious
  as it needs, and fix OPAM when needed} changes to OPAM, and required
management.

There are two distinct kinds of benchmarks, \emph{micro-benchmarks} (a specific
function) and \emph{macro-benchmarks} (a program).

\begin{itemize}
\item Benchmarks are specific OPAM packages, marked with a tag.
\item Don't change OPAM
\end{itemize}

\subsection{Description of a benchmark}

A benchmark is a specific OPAM package, it is not included in the
library or program package. Running is done by installing the
benchmark package. The install instruction copy the resulting data to
a directory fixed by convention. (probably
\texttt{\%\{prefix\}\%/bench/package\_name})

A package generating benchmark results should be annotated with the
\texttt{benchmark} tag.

\begin{itemize}
\item No or minimal change to opam are required.
\item Allow to add additionnal dependencies to the benchmark that are
  not required by the original package.
\end{itemize}

Potential problems:
\begin{itemize}
\item Benchmarking functions not available in a module interface is
  not directly possible. It can be circumvented by having a benchmark
  package containing the library sources (more package management
  work), or by exposing a benchmarkable value in the interfaces (can
  be done only by the library develloper).
\end{itemize}

\subsection{Benchmark result format}

Not decided, probably a simple json or csv like
format\footnote{Fabrice: it must be decided now. Any existing standard
  in other languages ?}

\section{Organisation in Sub-Projects}

Comment\footnote{Fabrice: we should divide the global project into
  several sub-projects, such as (1) local bench-marking a package, (2)
  local bench-marking the compiler, (3) community website and (4)
  OCaml performance weather server}

\subsubsection{System informations}

Recover as much relevant system informations as possible:
\begin{itemize}
\item Hardware informations: CPU, memory, ...
\item System informations: versions of OS, libraries, virtualisation, ...
\end{itemize}

\subsubsection{Macro-benchmark Runner}

Input:
\begin{itemize}
\item a binary path
\item expected result
\item variables, their encoding and potential values
\end{itemize}

Output:
\begin{itemize}
\item A benchmark file.
\item Additionnal informations to analyse the results provided by
  tools like \texttt{perf record} on linux or \texttt{callgrind}
\end{itemize}

Note: Reporting running time can be done simply in a cross platform
way.

Reporting more precise result like cycles is easy on linux but
other systems would require some search. GC statistics requires
changes to the original program. A possibility would be to addopt as
convention that the environment variable
\texttt{OCAML\_GC\_STATISTICS} contains a file where statistics should
be written (using Gc.print\_stat).

Defining a macrobenchmark package of a program built by an existing
opam package should not requires to provides additionnal sources (a
url file in the opam package).

\subsubsection{Microbenchmark runner}

Runs a set of functions, test the results, records the
timings/cycles/gc informations and write raw results to the directory
fixed by convention.

Probably the runner part of core\_bench. This should be extracted from
core\_bench (3 files) to minimize dependencies, especialy camlp4.

This should be provided as a template package, already containing
build system and base files. The user should only provide a .ml file
containing a list of functions to test and a file containing the list
of build dependencies (the ocamlfind -package part of the compilations
command).

This should allow to provide a microbenchmark as a simple opam package
containing those 2 files in the \texttt{files} subdirectory.

\subsubsection{Simple microbenchmark set}

A set of microbenchmarks able to run everywhere without any difference
should be provided to reliably test the compiler. It should not depend
on any kind of code generation, or feaure specific to a version.

\subsubsection{Management}

\begin{itemize}
\item List packages containing benchmarks
\item Prepare an opam switch to install a manualy built compiler
\item One command compile/run/recover/clean for a given opam compiler
  package and set of benchmarks.
\end{itemize}

\subsubsection{Agregation}

This tool should produce the set of graphs given a set of raw
results. It will probably use core\_bench for the statistics part. It
should produce the final html pages to display on the public site.

\subsubsection{Results management}

Use github: no tool needed.

Benchmark results should be managed in a centralised git repositories,
probably using github to minimize maintenance. Users publishing
results on their exotic systems should send patches/pull requests.

Common architectures results should always be provided by the same
computers to allow easier comparisons, all results should be
automaticaly produced and commited to the repository.

\section{Milestones}

Comment\footnote{Fabrice: we should define some initial milestones for
  every sub-project}

\end{document}
